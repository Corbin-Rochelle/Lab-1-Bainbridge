\documentclass[aps,prl,twocolumn]{revtex4-2}
% Please use this document class statement.  The prl option will help the formatting of your citations.

\usepackage{graphicx}
\newcommand{\etal}{\textit{et al.}}

% The following code is placed in the preamble of your document allowing me to directly add corrections and comments to your file.

\usepackage{xcolor}
\usepackage{circledsteps}
\newcommand{\Comment}[2]{\textcolor{red}{\CircledText{#1}#2}}
\newcommand{\rtext}[2]{\textcolor{red}{\sout{#1} #2}}
\usepackage{ulem}

\begin{document}

\title{Input title}

% Here is the complete author list.  Be sure to move your name to be first if you are the submitting author.

\author{D.~S.~Araya}
\affiliation{Department of Physics and Astronomy, Mississippi State University, Mississippi State, MS 39762, USA}

\author{B.~C.~Clark}
\affiliation{Department of Physics and Astronomy, Mississippi State University, Mississippi State, MS 39762, USA}

\author{K.~J.~Grimes}
\affiliation{Department of Physics and Astronomy, Mississippi State University, Mississippi State, MS 39762, USA}

\author{D.~C.~Heson}
\affiliation{Department of Physics and Astronomy, Mississippi State University, Mississippi State, MS 39762, USA}

\author{E.~L.~Lawson}
\affiliation{Department of Physics and Astronomy, Mississippi State University, Mississippi State, MS 39762, USA}

\author{J.~E.~Ratcliffe}
\affiliation{Department of Physics and Astronomy, Mississippi State University, Mississippi State, MS 39762, USA}

\author{C.~T.~Rochelle}
\affiliation{Department of Physics and Astronomy, Mississippi State University, Mississippi State, MS 39762, USA}

\author{M.~S.~Wright}
\affiliation{Department of Physics and Astronomy, Mississippi State University, Mississippi State, MS 39762, USA}

 \begin{abstract}
Input abstract. \end{abstract}
 \maketitle

\section{Introduction}
Your introduction.  Throughout the document do not forget to include citations.\cite{JJThomson}

\section{Theory}
Your theory section.

\section{Experiment}
A description of the experiment giving details.  You need to include a schematic of the device as well as details on how the measurement was made.

\section{Data Analysis}
Go over the analysis.  You should include plots of the data which will indicate the quality of the data.  The plot should include ``error bars'' and a fit line (not a trend line).  You should include tables of any values which are needed in the analysis but not represented by the information in the figures.

\section{Conclusion}
How does your result compare to the accepted value.  If you are within $2\sigma$ then the result is acceptable.

\begin{thebibliography}{9}
\bibitem{JJThomson} J.J. Thomson, Philosophical Magazine \textbf{44}, 293 (1897).
\end{thebibliography}

\end{document}
